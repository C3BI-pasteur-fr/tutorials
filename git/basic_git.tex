\definecolor{git}{RGB}{172, 49, 49}
\def\git#1{{\small\color{git}\$ git #1}\\}

\begin{frame}
    \begin{itemize}
        \item clone
        \item  checkout
        \item  add
        \item commit
        \item push
        \item pull
    \end{itemize}
\end{frame}

\begin{frame}[fragile]{Configuration}
	\begin{lstlisting}
$ git config --global user.name "First Last"
$ git config --global user.email "email@example.com"
$ git config --global color.diff "auto"
$ git config --global color.status "auto"
$ git config --global color.branch "auto"
	\end{lstlisting}
\end{frame}

\begin{frame}[fragile]{Create/Clone a repository}
	\begin{lstlisting}
$ # Clone a remote repository
$ git clone https://github.com/C3BI-pasteur-fr/tutorials.git |\pause|

$ # or create a local repository in the current folder
$ git init
	\end{lstlisting}
\end{frame}

\begin{frame}[fragile]{Write a new feature/bug fix}
	\begin{enumerate}
		\item Create a new branch
		\item Implement a new feature/Fix a bug
		\begin{enumerate}
			\item Step 1 towards new feature/bug fix
			\item Commit
			\item Step 2 towards new feature/bug fix
			\item Commit
			\item \ldots
		\end{enumerate}
		\item Merge this branch into master
		\item Delete this branch 
	\end{enumerate}
\end{frame}

\begin{frame}[fragile]{Step 1. Create a new branch locally}
	\begin{lstlisting}
$ # Which branches do we have?
$ git branch -a |\pause|
* master
remotes/origin/HEAD -> origin/master
remotes/origin/master
remotes/origin/services |\pause|

$ # Create a branch for our feature/bug fix...
$ git branch feature
$ # ... and switch to this branch
$ git checkout feature 
Switched to a new branch "feature" |\pause|

$ # or create + switch in one command
$ git checkout -b feature 
Switched to a new branch "feature"
  	\end{lstlisting}
\end{frame}

\begin{frame}[fragile]{Step 2. Implement a new feature/Fix a bug}
  	\begin{lstlisting}
$ # Add files to index (to-be-committed)
$ git add main.py __init__.py |\pause|

$ # Remove files from index
$ git reset HEAD __init__.py |\pause|

$ # Look at current index state
$ git status
On branch feature
Changes to be committed:
  (use "git reset HEAD <file>..." to unstage)

	modified:   main.py

Changes not staged for commit:
  (use "git add <file>..." to update what will be committed)
  (use "git checkout -- <file>..." to discard changes in working directory)

	modified:   __init__.py |\pause|

$ # Commit (in our local repository!)
$ git commit -m "Fixed TSS positions on - strand"
[feature fd07832] Fixed TSS positions on - strand
 1 file changed, 1 insertion(+)
  	\end{lstlisting}
\end{frame}


\begin{frame}[fragile]{Step 3. Merge the branch into master}
  	\begin{lstlisting}
# Switch to master
$ git checkout master
Switched to branch 'master' |\pause|


$ # Reapply our commits on the master branch
$ git rebase feature
First, rewinding head to replay your work on top of it...
Fast-forwarded master to feature.
	\end{lstlisting}
\end{frame}

\begin{frame}[fragile]{Step 4. Remove the branch}
  	\begin{lstlisting}
$ # Remove local branch
$ git branch -d feature
Deleted branch feature (was fd86490). |\pause|
	\end{lstlisting}
\end{frame}


\begin{frame}[fragile]{The scenario we've just seen}
	\begin{enumerate}
		\item Created a new branch \textbf{locally}
		\item Implemented a new feature/Fixed a bug
		\begin{enumerate}
			\item Step 1 towards new feature/bug fix
			\item Commit
			\item Step 2 towards new feature/bug fix
			\item Commit
			\item \ldots
		\end{enumerate}
		\item Applied the commits from this branch on the master \textbf{locally}
		\item Deleted this branch \textbf{locally} (it never existed elsewhere)
		\pause
		\item Push changes to the remote server (origin)
		\begin{lstlisting}
		$ git push origin master
		\end{lstlisting}
	\end{enumerate}
	\pause
	\textit{What if I am not the only one working on this feature/bug?}
\end{frame}

\begin{frame}[fragile]{Publishing the branch on origin}
	\begin{lstlisting}
$ # Push our branch 'feature' to a new branch 'feature' (will be created) on origin
$ git push origin feature:feature |\pause|

$ # From now on, to push (committed) changes we can...
$ git push origin feature |\pause|
	\end{lstlisting}
\end{frame}

\begin{frame}[fragile]{Getting updates from origin}
	\begin{lstlisting}
$ # Pull updates from origin
$ git pull |\pause|
     ...
There is no tracking information for the current branch.
Please specify which branch you want to merge with.
See git-pull(1) for details.

    git pull <remote> <branch> |\pause|
    
$ # Pull (what?) branch feature (from where?) from origin
$ git pull origin feature |\pause|

$ # We can also update config...
$ # (from where?)
$ git config branch.feature.remote origin
$ # (what?)
$ git config branch.feature.merge feature

$ # ...so that from now on, we can pull just by...
$ git pull
	\end{lstlisting}
\end{frame}


\begin{frame}[fragile]{Joining someone else's branch}
	\begin{lstlisting}
$ git checkout --track -b bug origin/bug
Branch bug set up to track remote branch refs/remotes/origin/bug.
Switched to a new branch "bug"
	\end{lstlisting}
	
	\pause
	\begin{tiny}
	{\color{eclipsePurple} Beware:} \textit{pull origin bug} would have merged the changes from the remote branch \textit{bug} into our current local branch!
\end{tiny}
\end{frame}

\begin{frame}[fragile]{Merge the branch into master}
  	\begin{lstlisting}
$ git checkout master
Switched to branch "master" |\pause|
$ git merge feature |\pause|

$ # Note that if the feature only existed locally, 
$ # we could have used rebase instead of merge:
$ git rebase feature
	\end{lstlisting}
\end{frame}


\begin{frame}[fragile]{Step 4. Remove the branch}
  	\begin{lstlisting}
$ # Remove local branch feature
$ git branch -d feature
Deleted branch feature (was fd86490). |\pause|


$ # Remove remote branch feature
$ git push origin :feature
- [deleted] feature
	\end{lstlisting}
\end{frame}



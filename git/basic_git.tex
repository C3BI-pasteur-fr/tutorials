\graphicspath{{images/}}


\begin{frame}[fragile]{The central and local repository}
  \includegraphics[height=0.9\textheight, width=\textwidth]{images/distributed}
\end{frame}


\begin{frame}[fragile]{Configuration}
	\begin{lstlisting}
$ git config --global user.name "First Last"
$ git config --global user.email "email@example.com"
$ git config --global color.diff "auto"
$ git config --global color.status "auto"
$ git config --global color.branch "auto"
	\end{lstlisting}
\end{frame}

\setbeamercovered{transparent=0}

\begin{frame}[fragile]{Create/Clone a repository}
	\begin{columns}
		\begin{column}{0.6\textwidth}
			\begin{lstlisting}
$ # Create a local repository in the current folder
$ git init |\pause|

$ # or clone a remote repository
$ git clone https://github.com/C3BI-pasteur-fr/tutorials.git
			\end{lstlisting}
		\end{column}
		\begin{column}{0.4\textwidth}
			\begin{center}
				\only<1> {
					\includegraphics[width=0.9\textwidth]{init.png}
				}\only<2> {
					\includegraphics[width=0.9\textwidth]{clone.png}
				}
			\end{center}
		\end{column}
	\end{columns}
\end{frame}


\begin{frame}[fragile]{First steps}
	\begin{columns}
		\begin{column}{0.6\textwidth}
			\begin{lstlisting}
|\pause|
$ # Create a file
$ touch README.txt  |\pause|

$ # Add files to index (to be committed)
$ git add README.txt |\pause|

$ # Look at the current index state 
$ git status
On branch master

Initial commit

Changes to be committed:
  (use "git rm --cached <file>..." to unstage)

	new file:   README.txt |\pause|

$ # Commit (to our local repository!)
$ git commit README.txt -m "Added README file"
[master (root-commit) b84669e] Added README file
 1 file changed, 0 insertions(+), 0 deletions(-)
 create mode 100644 README.txt |\pause|
 
$ # Push to the remote server
$ git push 
			\end{lstlisting}
		\end{column}
		\begin{column}{0.4\textwidth}
			\begin{center}
				\only<1> {
					\includegraphics[width=0.9\textwidth]{empty.png}
				}\only<2-4> {
					\includegraphics[width=0.9\textwidth]{touch.png}
				}\only<5> {
					\includegraphics[width=0.9\textwidth]{commit.png}
				}\only<6> {
					\includegraphics[width=0.9\textwidth]{push.png}
				}
			\end{center}
		\end{column}
	\end{columns}
\end{frame}


%\begin{frame}
%\includegraphics[height=0.9\textheight, width=\textwidth]{images/git_commit}
%\end{frame}


\begin{frame}{Everything is a branch!}
	\includegraphics[height=0.9\textheight, width=\textwidth]{gitflow}
\end{frame}

\subsection{One-user case}

\begin{frame}[fragile]{Implement a new feature/Fix a bug}
\begin{columns}
	\begin{column}{0.6\textwidth}
	\small
	\begin{enumerate}
		\item Create a new branch
		\item Implement a new feature/Fix a bug
		\begin{enumerate}
			\tiny
			\item Step 1 towards new feature/bug fix
			\item Commit
			\item Step 2 towards new feature/bug fix
			\item Commit
			\item \ldots
		\end{enumerate}
		\item Update the master
		\item Delete this branch 
	\end{enumerate}
	\end{column}
	\begin{column}{0.4\textwidth}
		\begin{center}
			\includegraphics[width=0.9\textwidth]{branch.png}
		\end{center}
	\end{column}
\end{columns}
\end{frame}


\begin{frame}[fragile]{Step 1. Create a new branch locally}
\begin{columns}
	\begin{column}{0.6\textwidth}
	\begin{lstlisting}
$ # Which branches do we have?
$ git branch -a
* master
remotes/origin/HEAD -> origin/master
remotes/origin/master
remotes/origin/services |\pause|

$ # Create a branch for our feature/bug fix...
$ git branch feature
$ # ... and switch to this branch
$ git checkout feature 
Switched to a new branch "feature" |\pause|

$ # or create + switch in one command
$ git checkout -b feature 
Switched to a new branch "feature"
  	\end{lstlisting}
  	\end{column}
	\begin{column}{0.4\textwidth}
		\begin{center}
			\only<1> {
				\includegraphics[width=0.9\textwidth]{branch_a.png}
			} \only<2-> {
				\includegraphics[width=0.9\textwidth]{branch_created.png}
			}
		\end{center}
	\end{column}
\end{columns}
\end{frame}


\begin{frame}[fragile]{Step 2. Implement a new feature/Fix a bug}
\begin{columns}
	\begin{column}{0.6\textwidth}
	\begin{lstlisting}
$ # Add files to index (to-be-committed)
$ git add main.py __init__.py |\pause|

$ # Look at current index state
$ git status
On branch feature
Changes to be committed:
  (use "git reset HEAD <file>..." to unstage)

	modified:   main.py

Changes not staged for commit:
  (use "git add <file>..." to update what will be committed)
  (use "git checkout -- <file>..." 
	to discard changes in working directory)

	modified:   __init__.py |\pause|

$ # Commit (to our local repository!)
$ git commit -m "Fixed TSS positions on - strand"
[feature fd07832] Fixed TSS positions on - strand
 1 file changed, 1 insertion(+) 
  	\end{lstlisting}
  	\end{column}
	\begin{column}{0.4\textwidth}
		\begin{center}
			\only<1-2> {
				\includegraphics[width=0.9\textwidth]{branch_created.png}
			} \only<3-> {
				\includegraphics[width=0.9\textwidth]{branch_commit.png}
			}
		\end{center}
	\end{column}
\end{columns}
\end{frame}


\begin{frame}[fragile]{Steps 3-4. Update the master and delete the branch}
\begin{columns}
	\begin{column}{0.6\textwidth}
  	\begin{lstlisting}
# Switch to master
$ git checkout master
Switched to branch 'master' |\pause|

$ # Reapply our commits on the master branch
$ git rebase feature
First, rewinding head to replay your work on top of it...
Fast-forwarded master to feature. |\pause|

$ # Delete local branch
$ git branch -d feature
Deleted branch feature (was fd86490). |\pause|

$ # Push changes to the remote server (origin)
$ git push origin master
	\end{lstlisting}
	\end{column}
	\begin{column}{0.4\textwidth}
		\begin{center}
			\only<1> {
				\includegraphics[width=0.9\textwidth]{branch_committed.png}
			}\only<2> {
				\includegraphics[width=0.9\textwidth]{branch_rebase.png}
			}\only<3> {
				\includegraphics[width=0.9\textwidth]{branch_delete.png}
			}\only<4> {
				\includegraphics[width=0.9\textwidth]{branch_pushed.png}
			}
		\end{center}
	\end{column}
\end{columns}
\end{frame}

\setbeamercovered{transparent=30}

\begin{frame}[fragile]{The scenario we've just seen}

\subsection{Multiuser case}
\begin{columns}
\begin{column}{0.6\textwidth}
	\tiny
	\begin{enumerate}
		\item<2> Created a new branch \textbf{locally}
		\item<3-4> Implemented a new feature/bug fix
		\begin{enumerate}
			\tiny
			\item<3> Step 1 towards new feature
			\item<3> Commit
			\item<4> Step 2 towards bug fix
			\item<4> Commit
			\item<4> \ldots
		\end{enumerate}
		\item<5> \textbf{Reapplied} the commits from this branch on the master \textbf{locally}
		\item<6> Deleted this branch \textbf{locally} (it never existed elsewhere)
		\item<7> Pushed changes to the remote server (origin)
	\end{enumerate}
\end{column}
\begin{column}{0.4\textwidth}
	\begin{center}
			\only<1> {
				\includegraphics[width=0.9\textwidth]{branch_a.png}
			}\only<2> {
				\includegraphics[width=0.9\textwidth]{branch_created.png}
			}\only<3> {
				\includegraphics[width=0.9\textwidth]{branch_commit.png}
			}\only<4> {
				\includegraphics[width=0.9\textwidth]{branch_committed.png}
			}\only<5> {
				\includegraphics[width=0.9\textwidth]{branch_rebase.png}
			}\only<6> {
				\includegraphics[width=0.9\textwidth]{branch_delete.png}
			}\only<7> {
				\includegraphics[width=0.9\textwidth]{branch_pushed.png}
			}\only<8> {
				\includegraphics[width=0.9\textwidth]{branch_many.png}
			}
	\end{center}
\end{column}
\end{columns}

\visible<8>{
	\begin{center}
	
	\tiny
	\textit{What if I am not the only one working on this feature/bug?}
	\end{center}
}
\end{frame}

\setbeamercovered{transparent=0}

\begin{frame}[fragile]{Publish the branch on origin}
\begin{columns}
	\begin{column}{0.4\textwidth}
	\begin{lstlisting}
	|\pause|
$ # Push our branch 'feature' 
$ # to a new branch 'feature' 
$ # (will be created) on origin
$ git push origin feature:feature |\pause|

$ # From now on, to push (committed) changes...
$ git push origin feature
	\end{lstlisting}
	\end{column}
	\begin{column}{0.6\textwidth}
		\begin{center}
			\only<1> {
				\includegraphics[width=.9\textwidth]{multiuser_local_branch.png}
			}\only<2> {
				\includegraphics[width=.9\textwidth]{multiuser_remote_branch.png}
			}\only<3> {
				\includegraphics[width=.9\textwidth]{multiuser_push_branch.png}
			}
		\end{center}
	\end{column}
\end{columns}
\end{frame}


\begin{frame}[fragile]{Join someone else's branch}
\begin{columns}
	\begin{column}{0.4\textwidth}
	\begin{lstlisting}
$ git checkout --track -b bug origin/bug
Branch bug set up to track remote branch 
	refs/remotes/origin/bug.
Switched to a new branch "bug"
	\end{lstlisting}
	\visible<2->{
	\begin{tiny}
	{\color{eclipsePurple} Beware:} \textit{pull origin bug} would have merged the changes from the remote branch \textit{bug} into our current local branch!
	\end{tiny}
	}
	\end{column}
	\begin{column}{0.6\textwidth}
		\begin{center}
			\includegraphics[width=.9\textwidth]{multiuser_track.png}
		\end{center}
	\end{column}
\end{columns}
\end{frame}


\begin{frame}[fragile]{Get updates from origin}
\begin{columns}
	\begin{column}{0.4\textwidth}
	\begin{lstlisting}
$ # Pull updates from origin
$ git pull |\pause|
     ...
There is no tracking information for the current branch.
Please specify which branch you want to merge with.
See git-pull(1) for details.

    git pull <remote> <branch> |\pause|
    
$ # Pull (what?) branch feature 
$ # (from where?) from origin
$ git pull origin feature |\pause|

$ # We can also update config...
$ # (from where?)
$ git config branch.feature.remote origin
$ # (what?)
$ git config branch.feature.merge feature

$ # ...so that from now on, 
$ # we can pull just by...
$ git pull
	\end{lstlisting}
	\end{column}
	\begin{column}{0.6\textwidth}
		\begin{center}
			\only<1-2> {
				\includegraphics[width=.9\textwidth]{multiuser_they_push.png}
			}\only<3-> {
				\includegraphics[width=.9\textwidth]{multiuser_pull.png}
			}
		\end{center}
	\end{column}
\end{columns}
\end{frame}


\begin{frame}[fragile]{Update the master}
\begin{columns}
	\begin{column}{0.4\textwidth}
  	\begin{lstlisting}
$ # Switch to master
$ git checkout master
Switched to branch "master" |\pause|

$ # Merge
$ git merge feature |\pause|

$ # Do not forget to push
$ git push
	\end{lstlisting}
	\end{column}
	\begin{column}{0.6\textwidth}
		\begin{center}
			\only<1> {
				\includegraphics[width=.9\textwidth]{multiuser_my_many_commits_pushed.png}
			}\only<2> {
				\includegraphics[width=.9\textwidth]{multiuser_my_merge.png}
			}\only<3-> {
				\includegraphics[width=.9\textwidth]{multiuser_remote_merge.png}
			}
		\end{center}
	\end{column}
\end{columns}
\end{frame}


\begin{frame}[fragile]{Remove the branch}
\begin{columns}
	\begin{column}{0.4\textwidth}
  	\begin{lstlisting}
$ # Remove local branch feature
$ git branch -d feature
Deleted branch feature (was fd86490). |\pause|


$ # Remove remote branch feature
$ git push origin :feature
- [deleted] feature
	\end{lstlisting}
	\end{column}
	\begin{column}{0.6\textwidth}
		\begin{center}
			\only<1> {
				\includegraphics[width=.9\textwidth]{multiuser_my_delete.png}
			}\only<2> {
				\includegraphics[width=.9\textwidth]{multiuser_remote_delete.png}
			}
		\end{center}
	\end{column}
\end{columns}
\end{frame}

\setbeamercovered{transparent=30}

\begin{frame}[fragile]{The scenario we've just seen}
\begin{columns}
\begin{column}{0.5\textwidth}
	\tiny
	\begin{enumerate}
		\item<1-2> Created a new branch locally \\\textbf{and published it on the remote origin}
		\item<3-4> Implemented a new feature/Fixed a bug
		\begin{enumerate}
			\tiny
			\item<3> Step 1 towards new feature/bug fix
			\item<3> Commit \textbf{(and maybe push to origin)}
			\item<3> \ldots
			\item<4> \textbf{Pull others' changes from origin}
			\item<4> \ldots
		\end{enumerate}
		\item<5> \textbf{Merged} this branch with the master \textbf{locally}
		\item<6> \textbf{Pushed changes to origin}
		\item<7-> Deleted this branch locally \textbf{and on origin}
	\end{enumerate}
\end{column}
\begin{column}{0.6\textwidth}
	\begin{center}
			\only<1> {
				\includegraphics[width=0.9\textwidth]{multiuser_local_branch}
			}\only<2> {
				\includegraphics[width=0.9\textwidth]{multiuser_remote_branch}
			}\only<3> {
				\includegraphics[width=0.9\textwidth]{multiuser_push_branch}
			}\only<4> {
				\includegraphics[width=0.9\textwidth]{multiuser_pull.png}
			}\only<5> {
				\includegraphics[width=0.9\textwidth]{multiuser_my_merge.png}
			}\only<6> {
				\includegraphics[width=0.9\textwidth]{multiuser_remote_merge.png}
			}\only<7> {
				\includegraphics[width=0.9\textwidth]{multiuser_my_delete.png}
			}\only<8> {
				\includegraphics[width=0.9\textwidth]{multiuser_remote_delete.png}
			}
	\end{center}
\end{column}
\end{columns}
%	\pause
%	\textit{What if I want to work in several branches in parallel?}
\end{frame}


\begin{frame}{Summary}
	\begin{tiny}
    \begin{itemize}
        \item[] {\color{eclipseBlue}git init} -- create an empty repository
        \item[] {\color{eclipseBlue}git clone <repository>} -- clone a repository into a new directory
        \item[]
        \item[] {\color{eclipseBlue}git branch -a} -- list existing branches
        \item[] {\color{eclipseBlue}git branch <branch>} -- create a branch
        \item[] {\color{eclipseBlue}git checkout <branch>} -- switch to the branch 
        \item[]
        \item[] {\color{eclipseBlue}git status} -- show the working tree status
        \item[] {\color{eclipseBlue}git add <filepath>} -- add a file to the index (to be committed)
        \item[] {\color{eclipseBlue}git commit -m <message>} -- record changes to the local repository
        \item[]
        \item[] {\color{eclipseBlue}git pull <from where?> <which branch?>} -- fetch and integrate changes
        \item[] {\color{eclipseBlue}git push <where?> <to which branch?>} -- push the changes the remote repository
        \item[]
        \item[] {\color{eclipseBlue}git merge <branch>} -- merge the selected branch into the current branch
        \item[] {\color{eclipseBlue}git rebase <branch>} -- reapply commits from the branch on top of the current branch
%        \item[]
%        \item[] {\color{eclipseBlue}git stash} -- record the current state of the working directory and the index, and clean the working directory
%        \item[] {\color{eclipseBlue}git stash list} -- list the stashes that currently exist
%        \item[] {\color{eclipseBlue}git stash apply <stash>} -- return to the state recorded in this stash
        \item[]
        \item[] {\color{eclipseBlue}git log [-{}-graph]} -- see commit logs
        \item[]
        \item[] {\color{eclipseBlue}git config [-{}-global] <key> <value>} -- set config option
    \end{itemize}
    
    \begin{center}
    	Detailed docs: \url{https://git-scm.com/docs}
    \end{center}
    \end{tiny}
\end{frame}

\definecolor{git}{RGB}{172, 49, 49}
\def\git#1{{\small\color{git}\$ git #1}\\}

\begin{frame}[fragile]{Why not just local git?}
  \begin{itemize}
  \item<1-> Local git alone via command line: Ok
  \item<2-> But lack of functionalities:
    \begin{itemize}
    \item User/permission management
    \item Bug tracking
    \item Duscussing implementations
    \item Forking other projects
    \item Easy collaborative development 
    \end{itemize}
  \item<3-> Several services were developed around git
    \begin{itemize}
    \item \includegraphics[width=15px]{images/GitHub-Mark.png} GitHub \url{https://github.com/}
    \item \includegraphics[width=15px]{images/bitbucket.png} bitbucket \url{https://bitbucket.org/}
    \item \includegraphics[width=15px]{images/GitLab_Logo.png} GitLab \url{https://git.framasoft.org/}
    \item ...
    \end{itemize}
  \end{itemize}
\end{frame}

\begin{frame}{GitHub}
  \begin{itemize}
  \item url : https://GitHub.com/ 
  \item Wikipedia: GitHub is a web-based Git repository hosting service. It offers all of the distributed revision control and source code management (SCM) functionality of Git as well as adding its own features. Unlike Git, which is strictly a command-line tool, GitHub provides a Web-based graphical interface and desktop as well as mobile integration. It also provides access control and several collaboration features such as bug tracking, feature requests, task management, and wikis for every project.
  \end{itemize}
\end{frame}

